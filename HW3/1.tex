\textbf{a)} We essentially repeat the argument for the single item, with agents bidding uniformly at random on the item they are supposed to get at the optimal allocation.

Let $x^*$ denote the optimal allocation. Let $A_i^*$ be the strategy of $i$-th player to bid uniformly at random $b_i\in[0,v_ij]$ on item $j$ s.t. $x_{ij}^*=1$. Then, for for any bids of other agents we have

$$
u_i(A^*,A_{-i}) \geq \sum_j x_{ij}^*[v_{ij} \frac{v_i-b_{max}}{v_ij} - v_{ij}/2]= \sum_j x_{ij}^*[v_{ij}/2-Rev_j(A)]
$$

Summing over all the agents yields

$$
\sum u_i(A_i^*,A_{-i}) \geq \sum_i x_{ij}^* v_{ij}/2 - Rev(A)
$$

Here we used the fact that $\sum_{i,j} x_{ij}^*Rev_j(A)\leq Rev(A)$ (revenue from some items is always smaller than the total revenue). The result now follows with the standard smoothness argument.

\textbf{b)} Can show that bidding on multiple items can be a best response, but is this enough?

\textbf{c)}

1. Notice that since $q_1=q_2$ the value of $q$ plays no role in utilities of bidders, essentially this is an all-pay auction in which both players have value of $p$ and highest bidder wins. As we know, the Nash equilibrium is to bid uniformly at random in $[0,p]$. (and in particular, bidding [0,1] is an equilibrium at p=1)

2. With two players and price $p$ the expected social welfare is $W=\frac{2}{3}pq-p/2-p/2=p(2q/3-1)$, and  welfare of auctioneer is the same $W=p(2/3q-p)$. So if $q>3/2$ auctioneer should set the prize as high as possible, and if $q<3/2$ he should set $p=0$. 