

\textbf{(a)}

Assume that we are in an equilibra, and given that some player $i\in I$ currently in queu $q \in Q$ with weight $W_q$ is not interested in unilateral deviation we must have

$$
W_q - w_i \le W_{q'} \qquad \forall q' \in Q
$$

summing this inequality and dividing by the number of queus gives us

$$
W_q - w_i \le \frac{1}{m} \sum_{q'} W_{q'}
$$

Now clearly we have $\frac{1}{m} \sum_{q'} W_{q'} \le OPT$ and $w_i \le OPT$, which gives us

$$
W_q \le 2 OPT
$$

If we let $i$ be the player with the maximal waiting time we have a bound on the POA.

\textbf{(b)}

We want both the inequalities in our proof above to essentially hold with equality. Thus we want one job $w_i$ to be equal to the optimal solution, and we want all ther queues to essentially be equal to the optima. A natural family is then one large job and a series of small jobs, WLOG we can let the small jobs have weight 1, and let there be $m^2-m$ jobs of them for $m$ machines. Consider the equilibrium where we place $m-1$ weight 1 jobs on every server, and then the large job on a server. Such a placement is an equilibrium with maximum makespan $2m -1$ if we let the large job have weight $m$. Then there is also a solution where $m-1$ servers each have $m$ jobs and one server has the large job is the optimal solution with a makespan of $m$. Clearly the POA approaches 2 as we let $m \rightarrow \infty$.



