\textbf{(a)} We show that a potential function does not exist with a counter example. Consider the case of 2 servers and 2 jobs with weights $w_1 = 1$ and $w_2 = 2$. Suppose originally both jobs are assigned to server $1$. Consider the following two scenarios:
\begin{enumerate}
\item First $1$ switches to second server, than $2$.
\item First $2$ switches to second server, than $1$.
\end{enumerate}
We now analyze the change in potential function using it's definition. Under scenario $1$ potential function changes by $-2+1=-1$, under scenario $2$ it changes by $-1+2=1$, thus the same configuration(both jobs assigned to $2$) has two different values of $\Phi$, which is a contradiction.

\textbf{(b)} Existence of NE is implied by (c). (if there is no NE, best response sequence never terminates).

NE does not have to be unique, slightly more interestingly nor does it's value (max load). Consider the following example: 2 servers, 3 jobs with weight $w=2$ and 2 jobs with weight $w=3$. Then configuration 2-2-2 and 3-3, configuration 2-3 and 22-3 are both NE.

\textbf{(c)} We argue that sequence of best responses is guaranteed to terminate at PNE. Let $x_i$ denote the load on server $i$. Consider the function $F(x) = \sum_i x_i^2$. 

\uline{Lemma.} When player switches according to her best response, $F(x)$ strictly decreases.

\textit{Proof.} Suppose player's weight is $w$, load on current server is $x_1+w$ and load on best response server is $x_2$. This implies $x_1>x_2$, which in turn implies 
$$(x_1+w)^2>x_1^2+(x_2+w)^2$$
(checks out algebraically)

\uline{Theorem} Sequence of best responses terminates.

\textit{Proof} By previous Lemma function $F(x)$ strictly decreases whenever best response happens (unless best response is to not move), and $F(x)$ is bounded below by $0$, hence the sequence terminates finitely.
