
\textbf{(a)}

\subsection*{a}

Consider the unilateral deviation where one player $i$ WLOG switches sides from $A$ to $B$. The change in his utility is the utility on $B$ minus the utility in $A$

$$
\sum_{j \in B} w_{ij} - \sum_{j \in A} w_{ij}
$$

If we consider the given potential function the change when player $i$ changes sides the change in the potential function is half of change of the players utility - calculated according to the above formula - plus the utility gained from all his friends in $B$ minus the utility lost from all his friends in $A$

$$
\frac{1}{2} \bigg( \sum_{j \in B} w_{ji} - \sum_{j \in A} w_{ji} \bigg)
$$

By inspection we see that the given formula is a potential for the game, when we use the \textbf{symmetry} $w_{ij} = w_{ji}$. 


\textbf{(b)}

Assume that we are in an equilibra, now for any given player $i$ in WLOG $A$, since $i$ does not want to move we must have that the utility of staying in $A$ being at least as large as that of swiching to $B$, thus

$$
\sum_{j \in A} w_{ij} \ge \sum_{j \in B} w_{ij}
$$

Now clearly the maximal possible utility $OPT_i$ of $i$ is bounded by $\sum_j w_{ij} = \sum_{j \in A} w_{ij} + \sum_{j \in B} w_{ij}$. Now clearly 

$$
\sum_{j \in A} w_{ij} \ge \frac{1}{2} OPT_i
$$

Since every player has at least half of their maximal utility, the sum over all players utility must be at least half of the maximum total utility.


\textbf{(c)}

We can define a simple congestion game with two congestible elements, one representing $A$ and the other representing $B$. The 'cost' of each element is simply minus the total utility of all players taking it. More formally, with $f_e$ being the set of players in element $e$, we take

$$
c_e(f_e) = - \sum_{i \in f_e} \sum_{j \in f_e} w_{ij}
$$

Now minimizing the congestion is the same as maximixing the utility.

\textbf{(d)}

When we do not assume symmetry the argument from $a$ fails. The game is no longer a potential game, consider a setup with just two players with assymetric friendship starting of in different groups, depending on what player moves the change in utility will vary - however any potential function will have to change in the same way no matter what player moves. 

The 2 bound on the POA will still hold if an equilibrium exists - just like before in an equilibra any player will enjoy at least half of their maximal utility.

We will give an example of best response dynamics that does not converge. Consider a triangle drama, where $a$ loves $b$, $b$ loves $c$ and $c$ loves $a$. Let us start with $a$ and $c$ in $A$ and $b$ in $B$. The best response of $a$ is to move to $B$, thereafter the best response of $b$ will be to move to $A$ and thereafter $c$ will move to $B$. This can be repeated without converging. 


