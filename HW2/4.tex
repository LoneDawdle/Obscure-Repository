
\textbf{a)} Formula $C=\sum_j R_j p_j$ is rather obvious, but let's demonstrate it formally. Let $x_{ij}$ be the indicator variable for job $j$ being assigned to server $i$, and let the ordering of jobs be such that $p_j$ is sorted in ascending order. We then have:

\begin{align*}
C = \sum_i \sum_{j=1}^m x_{ij}\sum_{k=1}^j x_{ik}p_k = \sum_i \sum_{k=1}^m p_k \sum_{j=k}^m x_{ij}x_{ik} = \sum_i \sum_{k=1}^m x_{ik} p_k R_k = \sum_k p_k R_k
\end{align*}

\textbf{b)}

\uline{\textbf{Lemma.}} For any allocation of jobs the value of $\sum R_j$ is less or equal to it's value at "j mod n" allocation. 

\textit{Proof.} Let $n_i=\sum_j x_{ij}$ be the number of jobs on server $i$. Then we have

\begin{align*}
\sum R_j = \sum_i \frac{n_i+1}{2} n_i = m/2 + \sum_i {n_i^2}{2}
\end{align*}

Now notice that the second term is minimized at "j mod n" allocation. \agcomment{why?}

\textbf{c)} Since we are in an equilibra we must have

$$
W_j(t_i) \le W_k(t_i) + t_i
$$

Summing these inequalities up over all queus $k$ gives

$$
W_j(t_i) \le \frac{\sum_k W_k(t_i)}{K} + t_i = \frac{\sum_{j: t_j < t_i}  t_j}{K} + t_i 
$$

The right hand equality comes from the fact that the total waiting time over all queus must be the sum of their waiting times, which is the sum of all their elements. This gives a bound on the waiting time for every task $i$. 

Let us assume that the total number of jobs is evenly divisible by $k$, which can be achieved by adding jobs of weight zero. Now in the given optimal solution the total waiting time for 'row' $q$ must be $K$ times their average waiting time plus the sum of the jobs, which is

$$
\sum_k \bigg( t_k + \frac{1}{k} \sum_{k'} W_{k'} \bigg) = \sum_k t_k +  \sum_{j: t_j < t_{qK}}  t_j 
$$

Now this is similar to our bound above, except for the limits of the summation. That difference can be bounded for any job $i$ with $t_i$ as the difference is at maximum the mean of the weights of the preceeding tasks in the same row as $i$, and as the jobs are sorted all preceeding jobs must have weight less then $t_i$. This gives 

$$
W_j(t_i) \le \sum_{j: t_j < t_{qK}}  t_j  + 2 t_i \le 2 \sum_{j: t_j < t_{qK}}  t_j  + 2 t_i = 2 W^*(t_i)
$$

If every jobs waits at most twice what it would wait in the optimal solution, the total cost is at most twice of that in the optimal solution.
