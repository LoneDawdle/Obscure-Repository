We now present a set of assumptions for which constant factor POA bound holds with correlated valuations. The proof was constructed close to the deadline and possibly there is some problem with it that we haven't been able to find.


Let $i^*$ denote the deviation where a player choses a single item to bid on according to the random distribution over what items he/she should get in the OPT, given his/her valuation. For notational clarity we assume that every player is independently given a random seed number $n_i \sim UNIF(0,1)$ upon which this randomized decision is deterministically based, we call the distribution over all players seed $N$. We additionally assume that the players in the equilibrium deterministically uses the seed to potentially randomize their strategy. We assume that the set $S_j$ denotes the random set of players going for item $j$ when they all use this deviation. We then make the following assumption

$$
VAR \bigg(\frac{1}{|S_j|+1} \bigg) = c_1^2 \mathbb{E} \bigg[ \frac{1}{|S_j|+1}   \bigg]^2
$$


$$
VAR \bigg(\sum_{i \in S_j} \frac{v_{ij}}{2}} \bigg) = c_2^2 \mathbb{E} \bigg[ \sum_{i \in S_j} \frac{v_{ij}}{2} \bigg]^2
$$


For some constants $c_1, c_2 < 1$. We also assume that there are a a large number of wortless items and that players that does not want to bid on ordinary items bids 0 on the worthless items. Ties are assumed to be broken lexicographically. First we start with the usual inequality

$$
\sum_i \mathbb{E}_{v \sim V} [ u_i] \ge  \sum_i \mathbb{E}_{ \substack{ v \sim V \\ n \sim N }} [ u_i(i^*, -i)] 
$$

Note that we use subscripts $n \sim N$ to denote the random seed $n_i$ upon which player $i$ bases his/her randomized deviation decision. If a player gets multiple items he/she will use the one with the highest value. We let $u_{ij}$ denote the value that play $i$ derives from item $j$, which will be $v_{ij}$ if the player gets the items and uses it, and zero else. We can now express the utility as the sum over items $j$

$$
= \sum_{ij} \mathbb{E}_{ \substack{ v \sim V \\ n \sim N }} [ u_{ij}(i^*, -i)]
$$


By linearity of expectations we can move the expecations out. The expectation is taken w.r.t to all players jointly drawn valuations and random seeds, and the utilities of individual players are for them using the deviation in an equilibra. We now interpret the players as using the deviation simultaneously, but getting the utilities based on the strategies the other players would use in the equlibrium given the random seed $n$ and their valuations.

$$
= \sum_j  \mathbb{E}_{ \substack{ v \sim V \\ n \sim N }} \bigg[ \sum_i   u_{ij}(i^*, -i)  \bigg]
$$


Now, since using the deviating strategy guarantees any player using it nonnegative utility for any item, if we just sum up the utilities $u_{ij}(i^*, -i)$ of a random subset of all utilities, we must get something that is not larger then all utilities. To chose this subset we now introduce the random indicator variable $x_{ij}$ that is 1 iff player $i$ bids on item $j$ and additionally has the highest value (or highest value and best lexicographic position among the highest bidders) among all players that bids on $j$ using the deviating strategy. Otherwise it is zero. We now have

$$
\ge \sum_j  \mathbb{E}_{ \substack{ v \sim V \\ n \sim N }} \bigg[ \sum_i  x_{ij} u_{ij}(i^*, -i)  \bigg]
$$

Now for any player $i$ we can use the fact that bidding half of your valuation $v_{ij}$ on an items $j$ guarantess the utility $v_{ij}/2 - rev(j)$ if you are just bidding only on $j$. For any player $i$ and item $j$ we have

$$
u_{ij} \ge 1_{\textrm{i bids only on j}} (v_{ij}/2 - rev(j))
$$

We multiply both sides of this equation by $x_{ij}$ which yields

$$
u_{ij} x_{ij} \ge 1_{\textrm{i bids only on j}} x_{ij} (v_{ij}/2 - rev(j))
$$

Now note that $x_{ij}=1 \implies 1_{\textrm{i bids only on j}} = 1$, since if a players is the highest bidder for $j$ it must also bid only on $j$ by design of the deviating strategy. Hence we have $x_{ij}= 1_{\textrm{i bids only on j}} x_{ij}$. Thus the bound $u_{ij} x_{ij} \ge x_{ij} (v_{ij}/2 - rev(j))$ holds for any random-seed vector and any set of valuations for all players, and we can use this bound within the expectation to get

$$
\ge \sum_j  \mathbb{E}_{ \substack{ v \sim V \\ n \sim N }} \bigg[ \sum_i  x_{ij} \bigg( \frac{v_{ij}}{2} - rev(j) \bigg) \bigg]
$$

Now as $x_{ij}$ is only one if player $i$ bids on $j$ with the deviating strategy and additionally has the highest value among all players bidding on $j$ when using the deviating strategy, we must have $\sum_{j} x_{ij} \le 1$ for every $j$. This is because there cannot be multiple deviating players all having the highest valuation for $j$ (and additionally best lexicographic position if there are ties). By linearity of expectations, and using the fact that the revenue for any item is non-negative, we now have

$$
=  \sum_j  \bigg( \frac{1}{2} \mathbb{E}_{ \substack{ v \sim V \\ n \sim N }} \bigg[ \sum_i  x_{ij} v_{ij} \bigg]  - \mathbb{E}_{ \substack{ v \sim V \\ n \sim N }} \bigg[ \sum_i x_{ij}  rev(j)  \bigg] \bigg)
$$


$$
\ge \sum_j  \bigg(  \frac{1}{2} \mathbb{E}_{ \substack{ v \sim V \\ n \sim N }} \bigg[ \sum_i  x_{ij} v_{ij} \bigg]  -   rev(j)  \bigg)
$$

We clearly here have the total revenue, which we can throw to the other side of the equation to get total utility. Let us define player $i^*_j$ for item $j$ as the player who bids on item $j$ when deviating, and additionally have the highest valuation for $j$ among deviating players bidding on $j$. If no player bids on $j$ then $i^*_j$ is undefined.  Let $v_{i^*_j j}$ be the valuation of item $j$ by $i^*$ if $i^*_j$ exits, and be zero else. By definition we have $\sum_j x_{ij} v_{ij} = v_{i^*_j j}$, hence the total utility $U$ we have

$$
U \ge  \frac{1}{2} \sum_j \mathbb{E}_{ \substack{ v \sim V \\ n \sim N }} \big[ v_{i^*_j j} } \big]
$$

Now the expected valuation $v_{i^*j}$ for item $j$ of the player with the highest valuation among deviating players bidding on $j$, cannot be smaller than the average valuation of $j$ among all players bidding on $j$. If there are no players bidding on item $j$ we have $v_{i^*j}=0$, but if we define define $\frac{1}{\| \emptyset\|} = 0$ we can write

$$
\frac{1}{2} \sum_j \mathbb{E}_{ \substack{ v \sim V \\ n \sim N }} \big[ v_{i^*_j j} } \big] \ge \sum_j \mathbb{E}_{ \substack{ v \sim V \\ n \sim N }} \bigg[ \frac{1}{|S_j|}    \sum_{i \in S_j} \frac{v_{ij}}{2} \bigg] \ge \sum_j \mathbb{E}_{ \substack{ v \sim V \\ n \sim N }} \bigg[ \frac{1}{|S_j|+1}    \sum_{i \in S_j} \frac{v_{ij}}{2} \bigg] 
$$

Recall that $S_j$ denotes the random set of players bidding on item $j$ when using the deviating strategy. Now using $\mathbb{E}[ab] = \bar{a} \bar{b} + COV(a,b)$ which holds for arbitrary random variables, we get the following expression

$$
= \sum_j \mathbb{E}_{ \substack{ v \sim V \\ n \sim N }} \bigg[ \frac{1}{|S_j|+1}   \bigg]  \mathbb{E}_{ \substack{ v \sim V \\ n \sim N }} \bigg[  \sum_{i \in S_j} \frac{v_{ij}}{2} \bigg]  + COV \bigg(\frac{1}{|S_j|+1}, \sum_{i \in S_j} \frac{v_{ij}}{2} \bigg)
$$


Now the correlation can be bounded by the product of the standard deviations of the two variables. Using our assumption we get

$$
\ge \sum_j \mathbb{E}_{ \substack{ v \sim V \\ n \sim N }} \bigg[ \frac{1}{|S_j|+1}   \bigg]  \mathbb{E}_{ \substack{ v \sim V \\ n \sim N }} \bigg[  \sum_{i \in S_j} \frac{v_{ij}}{2} \bigg] \bigg(1  - c_1 c_2 \bigg)
$$

Now, as the function $1/(x+1)$ is convex, from Jensen's inequality we get

$$
\mathbb{E} \bigg[ \frac{1}{|S_j|+1} \bigg] \ge \frac{1}{\mathbb{E} [|S_j|]+1} 
$$


Additionally, for a single item we have

$$
\mathbb{E} [|S_j|] = \sum_i \sum_{v_i \sim D} P(\textrm{i has j in OPT} | v_i) = \sum_i P(\textrm{i has j in OPT} ) \le 1
$$

Hence we get

$$
\mathbb{E} \bigg[ \frac{1}{|S_j|+1} \bigg] \ge \frac{1}{\mathbb{E} [|S_j|+1]} \ge \frac{1}{2}
$$


Using this fact we get

$$
\ge \frac{1}{2} \sum_j \mathbb{E}_{ \substack{ v \sim V \\ n \sim N }} \bigg[  \sum_{i \in S_j} \frac{v_{ij}}{2} \bigg] \bigg(1  - c_1 c_2 \bigg)
$$

Now as we assume that every player bids, however possibly on the 'worthless' items, the summation $\sum_j \sum_{i \in S_j}$ must go over all players, and hence we should have

$$
\ge \frac{1}{4} \bigg(1  - c_1 c_2 \bigg)  \sum_i\mathbb{E} [ v_{ij}]
$$

Now for a single player we show that $\mathbb{E} [ v_{ij}]$ is actually equal to $i$:s expected value in the optimal solution. If we let $D_{ij}$ denote the marginal distribution of valuations $v_{ij}$ we have

$$
\mathbb{E} [ v_{ij}] = \sum_{v_{ij} \sim D_{ij}} \sum_j v_{ij} P( \textrm{i gets j in OPT} \,  | \, v_{ij}) = \mathbb{E} [OPT_i]
$$

This gives us a $\frac{1}{4} \big(1  - c_1 c_2 \big) $ bound for POA.



