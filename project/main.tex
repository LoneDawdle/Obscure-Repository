
\documentclass[twoside,onecolumn]{article}

\usepackage{blindtext} % Package to generate dummy text throughout this template 

\usepackage[sc]{mathpazo} % Use the Palatino font
\usepackage[T1]{fontenc} % Use 8-bit encoding that has 256 glyphs
\linespread{1.05} % Line spacing - Palatino needs more space between lines
\usepackage{microtype} % Slightly tweak font spacing for aesthetics

\usepackage[english]{babel} % Language hyphenation and typographical rules
\usepackage{graphicx}
\usepackage[hmarginratio=1:1,top=32mm,columnsep=20pt]{geometry} % Document margins
\usepackage[hang, small,labelfont=bf,up,textfont=it,up]{caption} % Custom captions under/above floats in tables or figures
\usepackage{booktabs} % Horizontal rules in tables
\usepackage{subcaption}
\usepackage{lettrine} % The lettrine is the first enlarged letter at the beginning of the text
\usepackage{amsmath}
\usepackage{enumitem} % Customized lists
\setlist[itemize]{noitemsep} % Make itemize lists more compact

\usepackage{abstract} % Allows abstract customization
\renewcommand{\abstractnamefont}{\normalfont\bfseries} % Set the "Abstract" text to bold
\renewcommand{\abstracttextfont}{\normalfont\small\itshape} % Set the abstract itself to small italic text
\usepackage[ruled]{algorithm2e} % For algorithms
\renewcommand{\algorithmcfname}{ALGORITHM}
\SetAlFnt{\small}
\SetAlCapFnt{\small}
\SetAlCapNameFnt{\small}
\SetAlCapHSkip{0pt}
\IncMargin{-\parindent}
\usepackage{titlesec} % Allows customization of titles
\titleformat{\section}[block]{\large\scshape\centering}{\thesection.}{1em}{} % Change the look of the section titles
\titleformat{\subsection}[block]{\large}{\thesubsection.}{1em}{} % Change the look of the section titles

\usepackage{fancyhdr} % Headers and footers
\pagestyle{fancy} % All pages have headers and footers
\fancyhead{} % Blank out the default header
\fancyfoot{} % Blank out the default footer
\fancyhead[C]{} % Custom header text
\fancyfoot[RO,LE]{\thepage} % Custom footer text

\usepackage{titling} % Customizing the title section
\usepackage{amsthm}
\usepackage{hyperref} % For hyperlinks in the PDF
\usepackage{algorithmic}
\newcommand{\nc}{\newcommand}
\nc{\nt}{\newtheorem}
\nt{thm}{Theorem}[section]
\nt{cor}[thm]{Corollary}
\nt{prop}[thm]{Proposition}
\nt{lem}[thm]{Lemma}
\nt{defn}[thm]{Definition}
\nt{rem}[thm]{Remark}
\nt{exa}[thm]{Example}
\nt{ass}[thm]{Assumption}
\nc{\pf}{\noindent{\bf Proof\ \ }}


%----------------------------------------------------------------------------------------
%	TITLE SECTION
%----------------------------------------------------------------------------------------

\setlength{\droptitle}{-4\baselineskip} % Move the title up

\pretitle{\begin{center}\Huge\bfseries} % Article title formatting
\posttitle{\end{center}} % Article title closing formatting
\title{Article Title} % Article title
\author{%
\textsc{John Smith}\thanks{A thank you or further information} \\[1ex] % Your name
\normalsize University of California \\ % Your institution
\normalsize \href{mailto:john@smith.com}{john@smith.com} % Your email address
%\and % Uncomment if 2 authors are required, duplicate these 4 lines if more
%\textsc{Jane Smith}\thanks{Corresponding author} \\[1ex] % Second author's name
%\normalsize University of Utah \\ % Second author's institution
%\normalsize \href{mailto:jane@smith.com}{jane@smith.com} % Second author's email address
}
\date{\today} % Leave empty to omit a date
\renewcommand{\maketitlehookd}{%

}

\author{Johan Bjork, Artur Gorokh}

\begin{document}
\title{POA guarantees for correlated values
}
\maketitle
\section{Introduction}

\cite{survey} provides guarantees for the independent case, \cite{roughgarden} shows that POA can be arbitrary bad for general correlation. However correlation is practically important scenario, and even partial positive results can be of value.  

We consider two specific ideas: general per-item correlation and negative correlation. For first we show a counterexample, for second we outline a potential approach for a proof.

We also give an alternative proof of POA for uncorrelated values which we have discovered along the way and find to be more intuitive that the approach taken in \cite{survey}.

\section{General per-item correlation}

The counter-example in \cite{roughgarden} makes a significant use of inter-item correlation, i.e. $cor(v_{ij},v_{ik})\neq 0$. However it can be argued that such type of correlation is not as important in practice as per-item correlation where $v_{ij}$ and $v_{kj}$ are correlated but $cor(v_{ij},v_{ik}) = 0$. Unfortunately, it turns out that this additional assumption does not guarantee a constant factor POA, which we show with the following counter-example.

\section{Negative correlation}
The result of the previous section is surprising and somewhat discouraging. In this section we attempt to further strenthen the assumption and show that in case of negative per-item correlation (granted practically not as intersting), a constant POA bound holds. 

\section{Bayes-Nash POA for iid values via weak duality} 
We here provide a proof for the POA of 2 for Bayes-Nash equlibria we constructed in attempts to prove a result for correlated values. Although the approach did not prove to be of use for this purpose, we find it to be more intuitive than the proof of \cite{survey} and of independent interest.
%\begin{thebibliography}{9}
%\item[survey] 
%
%\item{2} Bhawalkar, Kshipra, and Tim Roughgarden. "Welfare guarantees for combinatorial auctions with item bidding." Proceedings of the twenty-second annual ACM-SIAM symposium on Discrete Algorithms. Society for Industrial and Applied Mathematics, 2011.
%\end{thebibliography}
%
\begin{thebibliography}{9}
\bibitem{survey} 
Price of anarchy in auctions survey.
 
\bibitem{roughgarden} 
 Bhawalkar, Kshipra, and Tim Roughgarden. "Welfare guarantees for combinatorial auctions with item bidding." Proceedings of the twenty-second annual ACM-SIAM symposium on Discrete Algorithms. Society for Industrial and Applied Mathematics, 2011.
 
\end{thebibliography}

\end{document}