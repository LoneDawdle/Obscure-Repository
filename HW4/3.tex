


\textbf{a)} Suppose the click-rates for setting the prices are $0$ and $1$ but that the real rates are $1$ and $1+\epsilon$, and that we have players who values clicks as $1$ and $1.5$. If both players report their real valuations the player with the higher valuation will pay $1$ per click, which gives him a utility of $0.5(1+\epsilon)$. If he changes strategy to report $1-\epsilon$ he will pay $0$, get $1$ click and have a utility of $1.5$.

We gave an example where bidding one's value wasn't dominating, in the same example bidding over ones value will not be dominating.

-todo, underbidding



\textbf{b)} We claim that no such  bound exist and it may be profitably to overbid with arbitrary $x$. 

Proof. We construct an example, intuition of which i, the following: if two estimated rates $\tilde \alpha '$ are virtually equal, while the actual rates are not, it makes the conpensation price between the spots close to zero, thus creating an incentive to overbid arbitrarily high.

More formally, suppose there are two bidders $v_1=\frac{1}{\delta}v_2$ and two spots, $\alpha_2=(1+\epsilon)\alpha_1$, s.t. $\alpha_2'=\alpha_1'+\delta^2$ (notice that $\delta$ might be arbitrarily small according to our assumptions). Now, for small enough $\delta$, if player 2 bids half his value, player 1 maximizes his utility by bidding $b_1=v_1/\delta$: $u_1(b_1=v_2) = (1+\epsilon)\alpha_1 v_1 - \delta^2 v_2$ (in contrast bidding truthfully yields $u_1(b_1=v_1) = \alpha_1 v_1$).

\textbf{c)}
%From the formula given we see that the utility a player receives is bounded by 

%$$
%(1+\epsilon) \bigg(\alpha_i u_i - \sum_{j>i}b_j (\alpha_{j-1} - \alpha_j) \bigg)
%$$

We claim it is always a dominated strategy to bid $b_i>v_i(\gamma-(1+\epsilon)^2)$

BWOC, suppose player bids $b_i=v_ix$ with $x>(1+\epsilon)$. Utility of such bid differs from utility of bidding $\tilde b_i=v_i(1+epsilon)$ only if there is another player bidding $b_j\in[v_i(1+\epsilon),v_ix]$ (and potentially multiple bidders in this interval). The gain bid $b_i=v_ix$ gives compared to $\tilde b_i$ is $v_i(\alpha_j-\alpha_i)$ where $j$ is the allocated spot from $b_i$. The increase in price compared to bidding $\tilde b_i$ compared lower-bounded with $\frac{\alpha_j}{\tilde \alpha_j}v_i(1+\epsilon)(\tilde \alpha_j-\tilde \alpha_i)$ (which corresponds to scenario in which only one other bidder bids exactly $v_i(1+\epsilon)$). Thus we have shown that the total gain in utility from bidding $b_i$ is

$$
\Delta u_i \leq v_i(\alpha_j-\alpha_i) - \frac{\alpha_j}{\tilde \alpha_j}v_ix( \alpha_j'- \alpha_i')
$$

Now let $\alpha_j=c \alpha_i$. This gives us $\alpha_j - \alpha_i =\alpha_i(c-1)$, $\alpha_j - \alpha_i \geq \alpha_i(\frac{c}{1+\epsilon} - (1+\epsilon))$. By assumption $c>1$, so $\Delta u_i$ is non-positive (which means bidding $vx$ is dominated) when $$ \leq v_i(\alpha_j-\alpha_i) \leq \frac{\alpha_j}{\tilde \alpha_j}v_ix( \alpha_j'- \alpha_i')$$

this in turn implies $x\geq (\gamma-(1+\epsilon)^2)$

\textbf{d)} This POA proof essentially repeats the that for simultaneous SP auction. At an equilibrium, consider the deviation of bidding your value.  Denote $max^j(b)$ the j-th maximum value in vector $b$. We have

$$
u_i(b_i^*,b_{-i})\geq \alpha_i^*(v_i - max^{i^*}b)
$$
Here $i^*$ is the spot player $i$ gets at optimal allocation. Now we can sum it up over all players, and because of no overbidding we have $\sum max^{i^*}b\leq SW(b)$. Moving this term to the left we get the POA of 2.

\textbf{e)} We have shown in c) that there exists a critical value $\bar x=(\gamma-(1+\epsilon)^2)$ s.t. bidding $b_i>v_i \bar x$ is always dominated. Hence at equilibrium $\sum_i b_i \leq \bar x SW(b)$. Now using the same approach as in d) we get the POA of $1+\bar x$

